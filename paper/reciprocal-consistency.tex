\section{Reciprocal Consistency}
\label{sec:recipr-cons}

In the property graph data model edges have direction, each edge having a pair of \emph{source} and \emph{destination} vertices. In the storage layer, edge information is stored with \textbf{both} the source and destination vertices. This allows an edge to be traversed in both directions at the same cost, facilitating improved query performance.

A common approach to storing graphs (arising from JanusGraph \cite{janusgraph} and TitanDB \cite{TitanDB}) is for database records to represent vertices containing both properties and an adjacency list containing edge \emph{pointers} to other vertices, Figure \ref{adj-list}. Consider, for example, the statement that Tolkien \textit{wrote} The Hobbit. It is expressed using vertices \emph{a} and \emph{b}, for Tolkien and The Hobbit respectively, and an edge \textit{wrote} running from \emph{a} (source) to \emph{b} (destination).

Using openCypher \cite{openCypher} this can be represented by:
\begin{Verbatim}[commandchars=\\\{\},fontsize=\small,xleftmargin=.2in]
\textcolor{blue}{MATCH} (a:\textcolor{green}{Person}), (b:\textcolor{green}{Book})
\textcolor{blue}{WHERE} a.\textcolor{red}{name} = 'Tolkien' \textcolor{blue}{AND} b.\textcolor{red}{title} = 'The Hobbit'
\textcolor{blue}{CREATE} (a)-[w:\textcolor{green}{WROTE}]->(b)
\end{Verbatim}

Adjacency lists of both \emph{a} and \emph{b} record information about the edge and this information is mutually reciprocal (or inverse) of each other: \emph{a}'s list will indicate `\emph{a} \emph{wrote} \emph{b}' while \emph{b}'s will have `\emph{b} \emph{written} by \emph{a}', Figure \ref{reciprocally-consistent-edge}. Thus, a query `list all titles by the author who wrote The Hobbit' can be answered starting at (destination vertex) \emph{b} and then traversing to (source vertex) \emph{a}, even though the edge is ``directed'' from \emph{a} to \emph{b} at model level abstraction. When the adjacency list entries for a given edge are mutually compatible like this, that edge is said to be \emph{reciprocally consistent}, a form of referential integrity. A query can read either the source or destination vertex and is able to reify the edge correctly, returning consistent results.

\begin{figure}[ht!]
  \centering
  \begin{tikzpicture}[node distance=2cm]
    \node (v1) [vertex2,xshift=0cm,yshift=0cm] {\small{\texttt{a:\textcolor{green}{Person}}}};

    \node (v2) [vertex2,xshift=5cm,yshift=0cm] {\small{\texttt{b:\textcolor{green}{Book}}}};

    \node [below of=v1,yshift=1cm] {\small{\texttt{\textcolor{red}{name}:Tolkien}}};
        \node [below of=v2,yshift=1cm] {\small{\texttt{\textcolor{red}{title}:The Hobbit}}};

    \draw [thick,->,>=stealth] (0.8,0)  -- node [midway,above] {:\textcolor{green}{\small{\texttt{WROTE}}}} (4.2,0);
  \end{tikzpicture}
  \caption{A reciprocally consistent edge.}
  \label{reciprocally-consistent-edge}
\end{figure}