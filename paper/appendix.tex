\section{Derivation of Conflict Probability}
\label{sec:deriv-confl-prob}

Under the \tDelta protocol each interleaving in Figure \ref{fig:conf} can be given a probability of occurring. Figures \ref{fig:a} and \ref{fig:b} are equivalent and can be summarized by the same probability. Letting $t_x = 0$, the probability that $T_x$ and $T_y$ conflict can be formulated as: $$ P \left[ ( T_x >  \Delta + D) \cap (T_y > \Delta-D) \right]$$ The arrival times of $T_x,T_y$ are assumed exponentially distributed, $T \sim \exp (\rho)$.
Where, $ \rho  = \frac{\lambda P_i}{2N_i}$, the probability a given operation accesses the incorrect record of a half-corrupted edge of type $i$.
The probability of $d$ exceeding $\Delta$ is exponentially distributed with mean $\mu$, $D\sim \exp(\mu)$.
Therefore,
\begin{dmath*}
   q^{new}_{i} =  {P \left[ ( T_1 >  \Delta + D) \cap (T_2 > \Delta-D)  \right]} \\
   =  \int_{0}^{\Delta}  \frac{\lambda P_i}{2 N_i} e^{-\frac{\lambda P_i}{2 N_i} d} e^{-\mu (\Delta+d)} e^{-\mu (\Delta-d)} dd + \int_{\Delta}^{\infty} \frac{\lambda P_i}{2 N_i} e^{-\frac{\lambda P_i}{2 N_i} d} e^{-\mu (\Delta+d)} dd  \\
   =  e^{-2 d \mu} - \left( \frac{\mu}{\frac{\lambda P_i}{2 N_i} + \mu} \right) e^{-(\frac{\lambda P_i}{2 N_i}+ 2\mu)d} \\
 \end{dmath*}
To simply the derivation of the conflict probability we assume FIFO which rules out the interleaving in Figure \ref{fig:c}.
 % To choose to bound $\Delta$ consider the probability of the message delay exceeding  $\Delta$.
 % \begin{align*}
 %   P \left[ M > \Delta \right] & = 1 - e^{- \delta \Delta} \\
 %   1 - e^{- \delta \Delta} & = 1 - \epsilon \\
 %   e^{- \delta d} & = \epsilon  \\
 %   \Delta & = - \frac{\ln(\epsilon)}{\delta}
 % \end{align*}

%  Substituting $\Delta$ into the above equation yields the conflict probability for a given $\epsilon$. For example, $ e = 0.001$ gives a $0.001$ \% probability the message delay exceeds $\Delta$, for this  $\Delta = 0.035s$
% \begin{align}
%   e^{- \lambda d} =  e^{-\delta d \frac{\lambda}{\delta}} = \epsilon^{\frac{\lambda}{\delta}} \label{eqn7}
% \end{align}
% Therefore, from (\ref{eqn5} and  (\ref{eqn7}),
% \begin{align}
%   & = \epsilon^2 - \frac{\delta}{\frac{\lambda P_i}{2 N_i} + \delta} e^{-\frac{\lambda P_i}{2 N_i} d} \epsilon^2 \\
%   & = \epsilon^2 \left[1 -\frac{\delta}{\frac{\lambda P_i}{2 N_i} + \delta} e^{-\frac{\lambda P_i}{2 N_i} d} \right]
% \end{align}


% Where, $ E \left[ T_1 \right] = E \left[ T_2\right] = \frac{1}{\delta}$.
% Therefore, $$ f(x) =\frac{\lambda P_i}{2N_i} e^{-\frac{\lambda P_i}{2N_i} x} $$

%  % =   \int_{0}^{d}  \frac{\lambda P_i}{2 N_i} e^{-\frac{\lambda P_i}{2 N_i} x} e^{-\delta 2 d} dx +  \int_{d}^{\infty} \frac{\lambda P_i}{2 N_i} e^{-\delta d} e^{-(\frac{\lambda P_i}{2 N_i} + \delta)x}  dx  \\
%  %        =   e^{-2 d \delta} \left[1 - e^{-\frac{\lambda P_i}{2 N_i} d} \right] + \frac{\frac{\lambda P_i}{2 N_i} e^{-\delta d}}{\frac{\lambda P_i}{2 N_i} + \delta} e^{-(\frac{\lambda P_i}{2 N_i} + \delta)d} \\
